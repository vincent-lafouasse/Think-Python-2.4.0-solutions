\begin{exercise}
\index{len function}
\index{function!len}

Write a function named \verb"right_justify" that takes a string
named {\tt s} as a parameter and prints the string with enough
leading spaces so that the last letter of the string is in column 70
of the display.

\begin{verbatim}
>>> right_justify('monty')
                                                                 monty
\end{verbatim}

Hint: Use string concatenation and repetition.  Also,
Python provides a built-in function called {\tt len} that
returns the length of a string, so the value of \verb"len('monty')" is 5.

\end{exercise}


\begin{exercise}
\index{function object}
\index{object!function}

A function object is a value you can assign to a variable
or pass as an argument.  For example, \verb"do_twice" is a function
that takes a function object as an argument and calls it twice:

\begin{verbatim}
def do_twice(f):
    f()
    f()
\end{verbatim}

Here's an example that uses \verb"do_twice" to call a function
named \verb"print_spam" twice.

\begin{verbatim}
def print_spam():
    print('spam')

do_twice(print_spam)
\end{verbatim}

\begin{enumerate}

\item Type this example into a script and test it.

\item Modify \verb"do_twice" so that it takes two arguments, a
function object and a value, and calls the function twice,
passing the value as an argument.

\item Copy the definition of 
\verb"print_twice" from earlier in this chapter to your script.

\item Use the modified version of \verb"do_twice" to call
\verb"print_twice" twice, passing \verb"'spam'" as an argument.

\item Define a new function called 
\verb"do_four" that takes a function object and a value
and calls the function four times, passing the value
as a parameter.  There should be only
two statements in the body of this function, not four.

\end{enumerate}

Solution: \url{http://thinkpython2.com/code/do_four.py}.

\end{exercise}



\begin{exercise}

Note: This exercise should be
done using only the statements and other features we have learned so
far.  

\begin{enumerate}

\item Write a function that draws a grid like the following:
\index{grid}

\begin{verbatim}
+ - - - - + - - - - +
|         |         |
|         |         |
|         |         |
|         |         |
+ - - - - + - - - - +
|         |         |
|         |         |
|         |         |
|         |         |
+ - - - - + - - - - +
\end{verbatim}
%
Hint: to print more than one value on a line, you can print
a comma-separated sequence of values:

\begin{verbatim}
print('+', '-')
\end{verbatim}
%
By default, {\tt print} advances to the next line, but you
can override that behavior and put a space at the end, like this:

\begin{verbatim}
print('+', end=' ')
print('-')
\end{verbatim}
%
The output of these statements is \verb"'+ -'" on the same line.
The output from the next print statement would begin on the next line.

\item Write a function that draws a similar grid
with four rows and four columns.

\end{enumerate}

Solution: \url{http://thinkpython2.com/code/grid.py}.
Credit: This exercise is based on an exercise in Oualline, {\em
    Practical C Programming, Third Edition}, O'Reilly Media, 1997.

\end{exercise}

