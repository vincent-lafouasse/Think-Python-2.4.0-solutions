\section{Exercises}

\begin{exercise}

It is a good idea to read this book in front of a computer so you can
try out the examples as you go.

Whenever you are experimenting with a new feature, you should try
to make mistakes.  For example, in the ``Hello, world!'' program,
what happens if you leave out one of the quotation marks?  What
if you leave out both?  What if you spell {\tt print} wrong?
\index{error message}

This kind of experiment helps you remember what you read; it also
helps when you are programming, because you get to know what the error
messages mean.  It is better to make mistakes now and on purpose than
later and accidentally.

\begin{enumerate}

\item In a print statement, what happens if you leave out one
of the parentheses, or both?

\item If you are trying to print a string, what happens if you
leave out one of the quotation marks, or both?

\item You can use a minus sign to make a negative number like
{\tt -2}.  What happens if you put a plus sign before a number?
What about {\tt 2++2}?

\item In math notation, leading zeros are ok, as in {\tt 09}.
What happens if you try this in Python?  What about {\tt 011}?

\item What happens if you have two values with no operator
between them?

\end{enumerate}

\end{exercise}



\begin{exercise}

Start the Python interpreter and use it as a calculator.

\begin{enumerate}

\item How many seconds are there in 42 minutes 42 seconds?

\item How many miles are there in 10 kilometers?  Hint: there are 1.61
  kilometers in a mile.

\item If you run a 10 kilometer race in 42 minutes 42 seconds, what is
  your average pace (time per mile in minutes and seconds)?  What is
  your average speed in miles per hour?

\index{calculator}
\index{running pace}

\end{enumerate}

\end{exercise}
